\hypertarget{index_intro_sec}{}\section{\-I\-N\-T\-R\-O\-D\-U\-C\-T\-I\-O\-N}\label{index_intro_sec}
\-This implementation of the \-S6a interface is meant to be used as a shared library. \-It contains a lot of \-A\-P\-Is that are useful to implement the s6a interface. \-It is bases on the freediameter(www.\-freediameter.\-net) software libraries 'linfdproto' and 'libfdcore' which contains the implementation for the diameter basic protocol and all other basic functionalities needed for this implementation to work. \-And it also requires the extensions 'dict\-\_\-dcca', 'dict\-\_\-dcca\-\_\-3gpp' and 'dict\-\_\-nasreq' found in freediameter to be loaded for this implementation to work.\hypertarget{index_architechture}{}\section{\-Architecture}\label{index_architechture}
\-The following diagram shows the general architecture  \hypertarget{index_init}{}\subsection{\-Initialize Diameter Module}\label{index_init}
 \hypertarget{index_install_sec}{}\section{\-I\-N\-S\-T\-A\-L\-A\-T\-I\-O\-N}\label{index_install_sec}
\hypertarget{index_step1}{}\subsection{\-S\-T\-E\-P 1 \-: I\-N\-S\-T\-A\-L\-L\-\_\-\-F\-R\-E\-E\-D\-I\-A\-M\-E\-T\-E\-R}\label{index_step1}
\-Since this library totally relies on freediameter libraries, freediameter software must be installed. \-To install freediameter follow the install instruction stated in the files found in the freediameter directory. \-The freediameter software in this directory should be used since it contains some modific codes.\hypertarget{index_step2}{}\subsection{\-S\-T\-E\-P 2 \-: B\-U\-I\-L\-D T\-H\-E S6a I\-N\-T\-E\-R\-F\-A\-C\-E I\-M\-P\-L\-E\-M\-E\-N\-T\-A\-T\-I\-O\-N L\-I\-B\-R\-A\-R\-Y}\label{index_step2}
\-Assuming it has been downloaded to you \-P\-C already. \-To build the \-S6a library, on your terminal goto the directory where this implementation is downloaded to, and excute the '\-Makefile', run the following on terminal\-:

\# make

\-This will create two new directories '\-O\-B\-J\-E\-C\-T\-S' and 'bin'. \-And the library is place in the new directory 'bin' under the name 'libssixa.\-so'. \-Use 'make debug' to build a library which gives some information where the error occurs.\hypertarget{index_testing}{}\section{\-T\-E\-S\-T\-I\-N\-G}\label{index_testing}
\-Testing for all functionalities provided by the library are not yet done. \-The testing approached used is end to end functionality test(i.\-e. message transfer from local peer node to remote peer node). \-While doing the end to end testing each function will also be tested since all the functions will be used when all the end to end testing is completed.

\-The end to end testing that has be done so far is for \-Update-\/\-Location-\/\-Request message and \-Update-\/\-Location-\/\-Answer messages.

\-Running the test\-:

mysql client and server should be installed on system where this test will be done.

\-To run the test, first the procedures mentioned in section 2.\-1. and section 2.\-2. must be completed successfuly. \-Then on your command terminal navigate to 'test' directory which is found in the main directory and excute '\-Makefile' and then excute 'copy.\-sh' as follows\-:

\# cd test \# make \# sh copy.\-sh

\-Then configure the password and username of the mysql server on your system in 'mysql.\-conf' file found in this directory('test' directory). \-Read the instruction in the file configure accordingly.

\-After this create the test \-H\-S\-S database by navigating to 'hss\-\_\-db' directory which is found in the current directory('test' directory) and excuting 'my.\-sh' as follows\-:

\# sh my.\-sh

\-Now navigate back to the 'test' directory and then navigate to 'testhss' directory and run the following command to simulate an \-H\-S\-S server which is waiting for \-S6a interface messages\-:

\# ./testapp hss

\-Then open another terminal and navigate to 'testmme1' directory which is found in the same directory as 'testhss'. \-From there run the following to simulate an \-M\-M\-E client sending \-Update-\/\-Location-\/\-Request and waiting for \-Upadate-\/\-Location-\/\-Answer\-:

\# ./testapp ulr peer1.\-localdomain

\-Now you should see the test running on both terminals.

\-Note \-: '\-Warnings' messages in the terminal are just notifications that an optional \-A\-V\-P is not present which is could be normal. \-The '\-E\-R\-R\-O\-R' messages immediatly following or preceding the 'warning' messages is caused by the same situation as the 'warning' message and it is generated by the freediameter application not the test application. 